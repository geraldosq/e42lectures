\section{In out process}

\subsection{Input/output in  C++}

\begin{verbatim}


Float_t AnalyzeSpill::EnterValue(){
  // How to input a number.

  //sprintf(text, "Please enter the value of %d", char_name_variable);

    string  input="";
    float input_value=0;

      cout << "Please enter the value :";
      getline(cin, input);
      // This code converts from string to number safely.
      stringstream(input) >> input_value ;
      cout << "You entered: " << input_value << endl;

      return input_value;

}
\end{verbatim}

\subsection{Input/output with file C++}

C++ provides the following classes to perfomr output and input of characters to/from files: 
http://www.cplusplus.com/doc/tutorial/files/

\begin{itemize}
\item ofstream: stream class to write on files
\item ifstream: stream class to read from files
\item fstream: stream class to both readn and write from/to files
\end{itemize}

These classes are deried directly from the classes istream and ostream. We have already uses objects whose types were theses classes: \verb|cin| is an object of class \verb|istream| and \verb|cout| is an object of class \verb|ostream|.

We can use our file stream the same way we are already used to use \verb|cin| and \verb|cout|, with the only difference that we have to associate these streams with physical files. 


\begin{verbatim}
#include <iostream>
#include <fstream>
using namespace std; 

int main (){

    ofstream myfile; 
    myfile.open("example.txt", ios::out | ios::app); 
     myfile << "Writing this to a file. \n";      
     myfile.close(); 
return 0; 


}

\end{verbatim}


\subsubsection{Closing a File }

We call the stream's member function \verb|close|. This member function takes flushes the associated buffers and closes the file: 

\begin{verbatim}
myfile.close(); 
\end{verbatim}

\subsubsection{Text files}

Text files stream are those where the ios::binary flag \textbf{is not} included in their opening mode. These file are designed to store text and thus all values that are input or putput from/to them can suffer some formatting transformations, which do not necesarrily correspond to their literal binary value. 


\begin{verbatim}
// writing on a text file
#include <iostream>
#include <fstream>
using namespace std;

int main () {
  ofstream myfile ("example.txt");
  if (myfile.is_open())
  {
    myfile << "This is a line.\n";
    myfile << "This is another line.\n";
    myfile.close();
  }
  else cout << "Unable to open file";
  return 0;
}
\end{verbatim}


Reading from a file can also be performed in the same way that we did with cin:


\begin{verbatim}
// reading a text file
#include <iostream>
#include <fstream>
#include <string>
using namespace std;

int main () {
  string line;
  ifstream myfile ("example.txt");
  if (myfile.is_open())
  {
    while ( getline (myfile,line) )
    {
      cout << line << '\n';
    }
    myfile.close();
  }

  else cout << "Unable to open file"; 

  return 0;
}

\end{verbatim}



This last example reads a text file and prints out its content on the screen. We have created a while loop that reads the file line by line, using getline. The value returned by getline is a reference to the stream object itself, which when evaluated as a boolean expression (as in this while-loop) is true if the stream is ready for more operations, and false if either the end of the file has been reached or if some other error occurred.
